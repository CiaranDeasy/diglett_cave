\documentclass[10pt,twoside,a4paper]{article}

\usepackage{a4}             % Adjust margins for A4 media

\usepackage{fancyhdr}

\renewcommand{\headrulewidth}{0.4pt}

\renewcommand{\footrulewidth}{0.4pt}

\fancyheadoffset[LO,LE,RO,RE]{0pt}

\fancyfootoffset[LO,LE,RO,RE]{0pt}

\pagestyle{fancy}

\fancyfoot{}

\fancyfoot[RO,LE]{\thepage\ / \pageref{LastPage}}

\fancyfoot[C]{\today}



\usepackage{lastpage}       % "n of m" page numbering

\usepackage{lscape}         % Makes landscape easier

\usepackage{portland}       % Switch between portrait and landscape

\usepackage{graphics}       % Graphics commands

\usepackage{wrapfig}        % Wrapping text around figures

\usepackage{epsfig}         % Embed encapsulated postscript

\usepackage{rotating}       % Extra graphics rotation

\usepackage{longtable}      % Page breaks within tables

\usepackage{supertabular}   % Page breaks within tables

\usepackage{multicol}       % Allows table cells to span cols

\usepackage{multirow}       % Allows table cells to span rows

\usepackage{texnames}       % Macros for common tex names

\usepackage{trees}          % Tree-like layout

\usepackage{float}			% Eliminate pesky floating figures

\usepackage{textcomp}		% Adds more math symbols.

\usepackage{array}          % Array environment

\usepackage{url}            % URL formatting

\usepackage{amsmath}        % American Mathematical Society

\usepackage{amssymb}        % Maths symbols

\usepackage{amsthm}         % Theorems

\usepackage{verbatim}       % Verbatim blocks
\usepackage{moreverb}
\let\verbatiminput=\verbatimtabinput

\usepackage{ifthen}         % Conditional processing in tex

\usepackage{xcolor}         % X11 colour names



\renewcommand{\oddsidemargin}{-20pt}

\renewcommand{\evensidemargin}{-20pt}

\renewcommand{\topmargin}{-30pt}

\renewcommand{\textwidth}{410pt}

\renewcommand{\marginparwidth}{100pt}



\setlength{\parindent}{0em}

\addtolength{\parskip}{1ex}



\begin{document}

\section*{Project Diglett Code Overview}

\subsection*{Introduction}

This document provides an \textbf{informal, non-exhaustive} description of the code layout of Project Diglett. It is intended to give potential employers a good enough idea of the code's structure and design to be able to evaluate it usefully.

\begin{figure}[H]
\centering \includegraphics[scale=0.5]{diagram.png}
\end{figure}

\subsection*{Basic structure}

The \verb|MainGameState| class represents the ``main'' state where the player is moving around and digging. On each frame, the \verb|gameTick()| method is called to take inputs and update the game state, and the \verb|draw()| method is called to draw the updated game state.

The class has a \verb|Player| object representing the player's status (health, inventory, etc) and containing the methods for moving the player.

There is also a \verb|World| object containing information about the game world, and a \verb|MainGUI| object containing drawable GUI panels to overlay the game world.

\subsection*{Input}

The \verb|MainInputHandler| class is defined within the \verb|MainGameState| class and contains a \verb|processInputs()| method that polls the window for input events and calls appropriate methods on \verb|MainGameState| based on what input was received.

\subsection*{World}

The \verb|World| class contains a \verb|WorldData| object encapsulating information about tiles in the game world. It contains a 2D array of \verb|Chunk|s, where each \verb|Chunk| is a $20 \times 20$ array of \verb|Tile|s. The ambition behind the potentially peculiar design is to be able to implement an infinite world, but this feature hasn't been developed yet.

There are also \verb|InteractiveEntity|s, of which there is currently only the \verb|Shop|.

\subsection*{Entry point}

The \verb|main()| method is in \verb|Game.cpp|. It creates a \verb|GameWindow| object and passes control to the \verb|mainLoop()| method in that object. \verb|GameWindow| extends the \verb|RenderWindow| class provided by the SFML library, which provides framerate control such that the main loop is repeated once every frame for as long as the game is running.

\subsection*{GameState stack}

The \verb|GameWindow| contains a stack of \verb|GameState| objects, and calls \verb|gameTick()| and \verb|draw()| on the top state every tick. During normal play, the \verb|MainGameState| will be active. Another state may be pushed if, for example, the player dies or the shop is opened. Pushing a new state lets us change the behaviour of the game, including how it handles inputs, as each \verb|GameState| will have its own \verb|InputHandler| specifying how inputs affect the game in that state.

\end{document}